\documentclass[a4 paper]{article}
\usepackage[utf8x]{inputenc}
\usepackage[spanish]{babel}

\begin{document}

El objetivo de estas notas es explicar los conceptos fundamentales de arquitectura de microcontroladores y organización de memoria para programar en C. Esto es necesario en los sistemas embebidos porque los recursos suelen ser escasos, y tener una buena representación de la memoria disponible ayuda a optimizar estos recursos.

\section{Arquitectura de Microcontroladores}

Recordemos el funcionamiento básico de los microcontroladores observando el diagrama en bloques básico de un microcontrolador, mostrado en la figura \ref{fig:arq_mic}.

\begin{figure}[h!]
  \centering
  \label{fig:arq_mic}
  \caption{Estructura interna básica de un microcontrolador}
\end{figure}

En el bloque ``Code Memory'' (memoria del programa) están almacenadas las instrucciones que ejecuta el microcontrolador desde el momento en que se resetea.

* La CPU es la unidad que realiza las instrucciones del programa. Todas las otras unidades (periféricos, memorias, etc.) se comunican con la CPU.
* Las instrucciones que realiza el CPU se guardan en la memoria de programa (Code Memory).
* Muchas instrucciones implican guardar datos temporales, los cuales se hacen en la Memoria de Datos.
* Tanto la CPU como otros bloques funcionales contienen almacenadores de memoria rápida llamados registros, optimizados según su ubicación.

CUANDO UNO HACE UN PROGRAMA, LO HACE PARA EL PROCESADOR! En el linker se hace una descripción de la memoria y del microcontrolador en particular pero el programa (es decir, el algoritmo y la función main) se hace para el procesador, no para el micro.

\end{document}
